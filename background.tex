\section{Background}

\subsection{Message Sequence Charts}


\subsection{Aspect-oriented Programming}

Aspect-oriented Programming (AOP)~\cite{Kiczales:2001} is a solution  to  address  problems  related  to  crosscutting  concerns   and   introduces   a   new   abstraction,   called   \textit{aspect} (which encapsulates typical AOP structures, i.e. joinpoints, pointcuts, advices and intertype declaraions),   as   well   as   a   flexible   mechanism   to   compose   aspects   with   object-oriented   components,   such  as  classes,  methods  and  attributes,  in  a  variety  of  ways.  Therefore,  this  new  paradigm  offers  means  to  separate  concerns  and,  consequently,  to  ensure  better  modularization.  %One  of  the  first  and  most  commonly  used AOP languages is AspectJ [2], which, as we have mentioned previously, is a direct extension of the Java programming language. For a more in-depth discussion about  AspectJ,  and  AOP  in  general,  the  reader  can  refer to [2]. 
